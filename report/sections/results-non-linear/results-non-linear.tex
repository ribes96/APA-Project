% 6. The results obtained using
%  at least two general non-linear methods (indicating
% the best set of parameters for each one); for both
%  classifcation and regression
% tasks, any of: one-hidden-layer MLP, the RBFNN, the SVM with RBF kernel, a
% Random Forest

\section{Results obtained using non-linear methods}

\subsection{Random Forest}

\red{
\begin{itemize}
    \item Puesto que random forest ya hace algún tipo de resampling, quizá era mejor probarlo con el dataset preprocesado original, y no con los 51 bags que hemos creado. Comentar por qué hemos hecho esto y lo que podría pasar
    \item Comentar los tipos de bosques que han quedado después del crossvalidation: la cantidad de árboles que tiene cada uno, el mínimo de instancias en cada nodo, la cantidad de atributos que se consideran, etc.
    \item Qué puede afectar a este modelo y cómo nos va a afectar a nosotros
    \item Mostrar los resultados y la confussion matrix
    \item Comentar los resultados
    \item Comentar las diferencias entre cada uno de los dos datasets
    \item Comentar por qué creemos que los resultados han sido buenos o malos
\end{itemize}
}

\subsection{Neural Network}

\red{
\begin{itemize}
    \item Ver en promedio cuantas neuronas había en cada modelo y cuanta regularización se ha usado, después de los datos de crossvalidation
    \item Indicar que no se han hecho skips, y que únicamente hay una capa oculta\ldots
    \item No hace ningún tipo de distinción entre los tipos de datos de entrada. Para él tozdo son reales, los booleanos también
    \item Qué puede afectar a este modelo y cómo nos puede afectar a nosotros
    \item Mostrar los resultados y la confussion matrix
    \item Comentar los resultados
    \item Comentar las diferencias entre cada uno de los dos datasets
    \item Comentar por qué creemos que los resultados han sido buenos o malos
\end{itemize}
}
