% 3. The relevant data exploration process (pre-processing, feature extraction/selection,
% clustering and visualization)


\section{Data exploration process}

\subsection{Pre-processing}
\subsubsection{Treatment of missing values}
Our dataset do not have missing values, so there is no need to treat them.
\subsubsection{Treatment of anomalous values}
\red{Quizá hay que quitar algunas personas por ser demasiado jóvenes comparadas con el resto}
\subsubsection{Treatment of incoherent values}
\todo[inline]{Redactar que FEV1 está mal, referenciar algún artículo que hable
sobre el tema y para justificar que está mal, decidir qué haremos con esos pacientes (eliminarlos, o inferir
sus valores de FEV1 en función de sus vecinos) y si los inferimos poner el proceso
como lo hemos hecho}
\red{El FEV1 tiene valores incoherentes. La mayoría están sobre 3, pero algunos
están sobre 60}
\begin{figure}[bh]
\centering
\includegraphics[width=10cm]{histFEV1}
\label{fig:histFEV1}
\caption{Show the ammount of people having each value}
\end{figure}
\subsubsection{Coding of non-continuous or non-ordered variables}
% \todo[inline]{El código que convierta nuestras variables categóricas en tiras
% de 0s y 1s}
\subsubsection{Possible elimination of irrelevant variables}

Some of the variables of our dataset are not well represented. In particular:

\begin{center}
\begin{tabular}{|c|c|}
  \hline
  \textbf{DGN} & There is just one patient with DGN = 1 and just 8 have DGN = 8 \\
  \hline
  \textbf{PAD} & Just 8 patients have PAD = True \\
  \hline
  \textbf{ASHTMA} & Just 2 patients have ASHTMA = True \\
  \hline
\end{tabular}
\end{center}

\todo[inline]{Redactar bien esta parte, indicando que puesto que tenemos pocos
datos, podemos permitirnos hacer varias ejecuciones, y que probaremos cada
combinación de eliminar y no eliminar cada una de estas variables y veremos
cual da mejores resultados con la cross-validation}

%
% \red{Algunas variables están muy poco representadas:}
% \red{Haremos los experimentos dos veces, uno con todos los datos originales,
% y otro quitando el atributo de MI, ASHTMA, DGN1 y DGN8, y entonces veremos cual
% da mejores resultados}
% \begin{itemize}
%   \item DGN: Solo un paciente tiene DGN1, y solo 2 tienen DGN8
%   \item Solo 2 pacientes tienen MI == True
%   \item Solo 8 pacientes tienen PAD == True
%   \item Solo 2 pacientes tienen ASHTMA == True
% \end{itemize}
\subsubsection{Creation of new useful variables (Feature extraction)}
\red{Entender cómo funciona MCA, y ver si podemos sacar una variable nueva}
\red{Quizá es interesante añadir la variable FEV/FEV1}
\subsubsection{Normalization of the variables}
We need to normalize only our numeric variables, which are the AGE, FEV and FV1.


\red{Solo se pueden normalizar FVC, FEV1 y AGE. Miraremos cuales dan mejores
resultados}
\subsubsection{Transformation of the variables}
Acording to \red{the paper we found} the accetable range for skewness in a numeric
variable is $(-2, +2)$. The skewness of our original variables AGE, FVC and FEV are:

\begin{center}
\begin{tabular}{| c | c |}
  \hline
  AGE & -0.1899413 \\
  FVC & 0.5417132 \\
  FEV1 & 5.597584 \\
  \hline
\end{tabular}
\end{center}

But we have to take into account that we've eliminated \red{22} patients, so the
new values are:

\todo[inline]{Poner los nuevos valores}

As the three variables are in the specified range, there is no need of
transforming them.

% \red{Skewness es la asimetría de los datos respecto a la media.}
% \red{Como la mayoría de nuestras variables son categóricas, no tiene mucho
% sentido medir el skewness, ni tampoco corregirlo}
% \red{Kurtosis igual que el skewness, no es necesario porque la mayoría con
% categóricas}
 % consultar si debemos ignorar este punto
% \todo[inline]{El código en r para corregir la asimetría (skewness) de los datos}
% \red{Es aceptable un skewness entre -2 y +2. AGE y FVC ya están en este rango, por
% lo tanto no hay que hacer nada con ellos. FEV1 sí que se sale, pero los datos que
% tiene son erroneos. Por lo tanto los transformatemos de alguna forma (eliminarlos
% o inferirlos) y miraremos el skewness de esos nuevos datos. (suponemos que entonces
% sí que estarán en ese rango, y por tanto no habrá que aplicar la transformación)}

\red{Referenciar (y leer un poco...) el paper}


%https://www.researchgate.net/publication/281345819_The_Research_Methods_Knowledge_Base


% \subsection{Feature extraction/selection}
\subsection{Clustering}
\red{hacer varios k-means con distintos valores de k (2,3,4,5,6) para ver si descubrimos algún cluster que nos permita crear una variable nueva}
\subsection{Visualization}
\red{Hacer MCA  }
