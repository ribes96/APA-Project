% 3. The relevant data exploration process (pre-processing, feature extraction/selection,
% clustering and visualization)
\section{Data exploration process}

\subsection{Pre-processing}
\subsubsection{Treatment of missing values}
Our dataset do not have missing values, so there is no need to treat them.
\subsection{Treatment of anomalous values}
\red{Quizá hay que quitar algunas personas por ser demasiado jóvenes comparadas con el resto}
\subsubsection{Treatment of incoherent values}
\todo[inline]{Mirar si tenemos valores incoherentes}
\red{El FEV1 tiene valores incoherentes. La mayoría están sobre 3, pero algunos
están sobre 60}
\subsubsection{Coding of non-continuous or non-ordered variables}
\todo[inline]{El código que convierta nuestras variables categóricas en tiras
de 0s y 1s}
\subsubsection{Possible elimination of irrelevant variables}
\red{Algunas variables están muy poco representadas:}
\red{Haremos los experimentos dos veces, uno con todos los datos originales,
y otro quitando el atributo de MI, ASHTMA, DGN1 y DGN8, y entonces veremos cual
da mejores resultados}
\begin{itemize}
  \item DGN: Solo un paciente tiene DGN1, y solo 2 tienen DGN8
  \item Solo 2 pacientes tienen MI == True
  \item Solo 8 pacientes tienen PAD == True
  \item Solo 2 pacientes tienen ASHTMA == True
\end{itemize}
\subsubsection{Creation of new useful variables (Feature extraction)}
\red{Miraremos de hacer LDA y veremos qué combinaciones lineales son más
interesantes. Después haremos los experimentos con y sin LDA y veremos cuál da
mejores resultados }
\subsubsection{Normalization of the variables}
\red{Solo se pueden normalizar FVC, FEV1 y AGE. Miraremos cuales dan mejores
resultados}
\subsubsection{Transformation of the variables}
\red{Skewness es la asimetría de los datos respecto a la media.}
\red{Como la mayoría de nuestras variables son categóricas, no tiene mucho
sentido medir el skewness, ni tampoco corregirlo}
\red{Kurtosis igual que el skewness, no es necesario porque la mayoría con
categóricas}
 % consultar si debemos ignorar este punto
\todo[inline]{El código en r para corregir la asimetría (skewness) de los datos}

\subsection{Feature extraction/selection}
\subsection{Clustering}
\red{hacer varios k-means con distintos valores de k (2,3,4,5,6) para ver si descubrimos algún cluster que nos permita crear una variable nueva}
\subsection{Visualization}
\red{Hacer PCA o LDA y hacemos algunos gráficos usando únicamente las
``dimensiones'' más relevantes, y quizá sale algo interesante }
