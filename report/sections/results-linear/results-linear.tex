% 5. The results obtained using
%  at least three linear/quadratic methods (indicating
% the best set of parameters for each one):
% (a) If the task is classifcation, any of:
%  logistic regression, multinomial regression
% (single-layer MLP), LDA, QDA, RDA, Naive Bayes, nearest-neighbours, linear
% SVM, quadratic SVM
% (b) If the task is
%  regression, any of:
%  linear regression, ridge regression, the LASSO,
% nearest-neighbours, linear SVM, quadratic SVM
%  at least two general non-linear
%
\section{Results obtained using linear/quadratic methods}


\subsection{Naive Bayes}
\todo[inline]{Buscar la biblioteca que lo calcula y aplicarlo a nuestros datos}
\subsection{KNN}


\subsection{LDA}
% \red{Usar LDA para tener 2 centroides, que son los de cada una de las clases.
% Cuando evaluamos un dato nuevo, le aplicamos la transformación y miramos si queda más cerca de uno o de otro.
% Así podemos ver la probabilidad de que pertenezca a cada una de las clases.}
\red{Suponiendo que las varianzas de cada una de las clases son la misma, se
usa este algoritmo, (que simplifica QLA) para ver la probabilidad de pertenencia
a una clase}
\subsection{QDA}
\subsection{RDA}

\subsection{Logistic Regression}





\red{Mirar el vecino más cercano para precedir}

\red{Si suponemos que las variables son independientes:
-Haces naive Bayes para ver la probabilidad de que pertenezca a cada una de las clases
(habría que estudiar si las variables son independientes)
- Logistic regression}
