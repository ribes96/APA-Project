% 5. The results obtained using
%  at least three linear/quadratic methods (indicating
% the best set of parameters for each one):
% (a) If the task is classifcation, any of:
%  logistic regression, multinomial regression
% (single-layer MLP), LDA, QDA, RDA, Naive Bayes, nearest-neighbours, linear
% SVM, quadratic SVM
% (b) If the task is
%  regression, any of:
%  linear regression, ridge regression, the LASSO,
% nearest-neighbours, linear SVM, quadratic SVM
%  at least two general non-linear
%
\section{Results obtained using linear/quadratic methods}


\subsection{Naive Bayes}
% \todo[inline]{Buscar la biblioteca que lo calcula y aplicarlo a nuestros datos}

Naive Bayes Algorithm has the advantage that it doesn't distinguish between types of data, and it doesn't perform any implicit transformation. The typical disadvantadge of this method is that it assumes independence on the attributes of the data.

As we have mixed data, the advantage is very appropriate. Looking at our data, it seems like most of the attributes are independent, so we can assume the results will be good. In fact, as we are working with a reduced dataset (in which we have removed some of the attributes), we could expect that the results will be even better in that one.

As this is method is so simple, is doesn't need parameters, so we don't event need the crossvalidation process and we don't have to chose any hyper-parameter.

The results obtained using this model are:

\red{Mosrar los resultados y la confussion matrix}

At the first sight, it stands out the fact that the results are very different from the ``original'' dataset and the ``removed'' one. While ``original'' shows results over \red{value}, the other one shows much better results. This leads us to think that the attributes we removed weren't indeed independent from the others, and keeping theme out has helped the model.

\red{
\begin{itemize}
    \item Puede ser adecuado porque admite tipos de datos distintos sin ningún problema
    \item No necesita ningún hiperparámetro
    \item Puede ser más sensible a haber quitado algunos atributos, pues afecta a la suposición Naive de independencia
    \item Mostrar la confussion matrix
    \item Comentar si muestra buenos o malos resutlados
    \item Comentar diferencias entre cada uno de los dos dataset, y explicar porq qué
    \item Identificar por qué suponemos que los resultados son buenos o malos.
\end{itemize}
}

% \red{
% \begin{itemize}
%     \item Comentar las decisiones que se han tomado (cuidado con los valores por defecto). p.e: Por qué hemos usado distancia Manhatan en vez de Euclidea en knn
%     \item Comentar los hiperparámetros que se han obtenido por crossvalidation. Como no vamos a hablar de cada uno de los modelos (51), hablar de la media de todos ellos, o algo así
%      \item Comentar los resultados que se ha obtenido. Intentar ver por qué ha ido bien o mal, y cosas así
%      \item Poner la confussion matrix
% \end{itemize}
% }
\subsection{KNN}

KNN will look for similar patients and will predict the majority among them. This method is very sensitive to very imbalanced data, since with a lot of patients in one class it is very probably that many of them will be very close to the positive cases. Hence, we hope that our resampling method will be suitable for this model, as it only trains it with a balanced dataset of patients.

Nearest Neighbours is also sensitive to non-normalized dataset. Although the method in R used states that is does it, we have also normalized the data, just in case.

\red{Comentar la distancia usada}

Looking at the models created with crossvalidation we see that the average number of neighbours considered is \red{value}. The results obtained using this model are:

\red{Mostrar los resultados y la confussion matrix}
\red{Comentar los resultados}

\red{
\begin{itemize}
    \item Por qué decidimos usar distancia manhattan en vez de euclidea
    \item Ver, en promedio, cuantos vecinos se han considerado
    \item Comentar que nosotros ya hemos normalizado los datos y que hemos puesto de forma adecuada las variables categóricas
    \item Qué podría afectar a este modelo y cómo nos afectará a nosotros.
    \item Mostrar los resultado y la confussion matrix
    \item Comentar los resultados
    \item Comentar las diferencias entre cada uno de los dos datasets
    \item Comentar por qué creemos que los resultados han sido buenos o malos
\end{itemize}
}

% \red{Si no hacemos nada para desbalancear los datos, hay que usar una k muy pequeña}

% \red{Quizá es recomendable usar algún método para balancear los datos, y probar así otros valores de k}


% \subsection{LDA}
% % \red{Usar LDA para tener 2 centroides, que son los de cada una de las clases.
% % Cuando evaluamos un dato nuevo, le aplicamos la transformación y miramos si queda más cerca de uno o de otro.
% % Así podemos ver la probabilidad de que pertenezca a cada una de las clases.}
% \red{Suponiendo que las varianzas de cada una de las clases son la misma, se
% usa este algoritmo, (que simplifica QLA) para ver la probabilidad de pertenencia
% a una clase}
% \subsection{QDA}
% \subsection{RDA}

\subsection{General Linear Model}

The main problem of using this model is that is doesn't distinguish between different data types. For this model all the data are real numbers. As most of our variables are categorical, we assume it will not reflect very well our data. In addition, the power of this method relies on the basis function that the user can define, according to the meaning he knows they may have. We haven't defined any, due to lack of medical knowledge, so this feature won't be used.

The average \(\lambda\)-value for the models after crossvalidation is \red{value}.

The results of running this model on the dataset are:

\red{Poner los resultados y la confussion matrix}
\red{Comentar los resultados y las diferencias}

\red{
\begin{itemize}
    \item Los datos se considera todos como reales, y no entiende de booleanos o categóricas
    \item No hemos usado funciones de base especiales, solamente la identidad con cada uno de los atributos.
    \item Ver en promedio cual ha sido la regularización
    \item Qué puede afectar a este modelo y cómo nos va a afectar a nosotros
    \item Mostrar los resultados y la confussion matrix
    \item Comentar los resultados
    \item Cometnar las diferencias entre cada uno de los datasets
    \item Comentar por qué creemos que los resultados han sido buenos o malos
\end{itemize}
}






% \red{Mirar el vecino más cercano para precedir}

% \red{Si suponemos que las variables son independientes:
% -Haces naive Bayes para ver la probabilidad de que pertenezca a cada una de las clases
% (habría que estudiar si las variables son independientes)
% - Logistic regression}
