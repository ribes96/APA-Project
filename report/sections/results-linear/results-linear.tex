% 5. The results obtained using
%  at least three linear/quadratic methods (indicating
% the best set of parameters for each one):
% (a) If the task is classifcation, any of:
%  logistic regression, multinomial regression
% (single-layer MLP), LDA, QDA, RDA, Naive Bayes, nearest-neighbours, linear
% SVM, quadratic SVM
% (b) If the task is
%  regression, any of:
%  linear regression, ridge regression, the LASSO,
% nearest-neighbours, linear SVM, quadratic SVM
%  at least two general non-linear
%
\section{Results obtained using linear/quadratic methods}

\subsection{LDA}
Usar LDA para tener 2 centroides, que son los de cada una de las clases.
Cuando evaluamos un dato nuevo, le aplicamos la transformación y miramos si queda más cerca de uno o de otro.
Así podemos ver la probabilidad de que pertenezca a cada una de las clases.

% TODO consultar a Lluís si LDA debería dar una varianza para cada cluster


Mirar el vecino más cercano para precedir

Si suponemos que las variables son independientes:
-Haces naive Bayes para ver la probabilidad de que pertenezca a cada una de las clases
(habría que estudiar si las variables son independientes)
- Logistic regression

LDA (Fisher) para reducir la dimensión y hacer los otros
