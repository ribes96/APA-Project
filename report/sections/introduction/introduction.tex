% 1. A brief but self-contained description of the work and its goals, and of the available
% data, and any additional information that you have gathered and used
\section{Introduction}

\subsection{Desciption of the work and its goals}

The goal of this project is to build a classification model to predict whether
a lung cancer patient will die within one year after surgery or not. To do so
we will study a dataset with real lung cancer patients.

As this is very sensitive information, our priority will be to minimize the
amount of false negatives, i.e., avoid predicting a patient will not die within
one year when he actually does.

% \red{Put link in the references instead of in the text?}
The data is taken from
% \url{https://archive.ics.uci.edu/ml/datasets/Thoracic+Surgery+Data#}
\cite{zieba2013boosted}

\subsection{Desciption of available data}

The data we are working with is about patients who underwent major lung
resections for primary lung cancer in the years from 2007 to 2011. For each
patient we are given information about his diagnosis and effects produced
by the cancer.

The dataset is very limited in the number of instances available as it only contains
data about 470 patients. In addition, the distribution of the predicted class is not quite
balanced,
since only 70 of the patients died in a one year period after their surgery. This may
become a problem in some of the prediction models due to the fact that the results
will be biased towards the bigger class. However, we can suppose that the data has been
collected uniformly and that this proportion is similar to the probability of the real population.

For each patient we have 16 different attributes of which three are numerical, and
the rest is categorical. From those, ten are binary variables. The response attribute is
also binary.


% \subsection{Needed packages}
%
% The following R packages have been used in this work:
%
% \begin{itemize}
% \item{klaR}
% \item{randomForest}
% \item{caret}
% \item{ROCR}
% \item{kknn}
% \end{itemize}
