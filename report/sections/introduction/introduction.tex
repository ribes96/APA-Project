% 1. A brief but self-contained description of the work and its goals, and of the available
% data, and any additional information that you have gathered and used
\section{Introduction}

\subsection{Desciption of the work and its goals}

The goal of this project is to build a classification model to predict whether
a lung cancer patient will die within one year after surgery or not. To do so
we will study a dataset with real lung cancer patients.

As this is very sensitive information, our priority will be to minimize the
amount of false negatives, i. e, avoid predicting a patient will not die within
one year when it certainly does.


The data is taken from
\url{https://archive.ics.uci.edu/ml/datasets/Thoracic+Surgery+Data#}
\cite{zieba2013boosted}

\subsection{Desciption of available data}

The data we are working with is about patients who underwent major lung
resections for primary lung cancer in the years from 2007 to 2011. For each
patient we are given information about his diagnosis and effects produced
by the cancer.

The dataset is very limited in the number of instances available: it only has
470. In addition, the distribution of the predicted class isn't quite balanced,
since only 70 of the patients died in one year period. This may become a problem
in some of the prediction models due to the fact that the results will be biased
towards the biggest class. However, we can suppose that the data has been
collected uniformly and that this proportion is similar to the real one.

For each patient we have 16 different atributes. 3 of them are numerical, and
the rest are categorical. From those, 10 are binary. The response atribute is
also binary.
