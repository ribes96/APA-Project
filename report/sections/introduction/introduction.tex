% 1. A brief but self-contained description of the work and its goals, and of the available
% data, and any additional information that you have gathered and used
\section{Introduction}

\subsection{Desciption of the work and its goals}

The goal of this project is to build a classification model to predict whether
a lung cancer patient will die within one year after surgery or not. To do so
we will study a dataset with real lung cancer patients.

As this is very sensitive information, our priority will be to minimize the
amount of false negatives, i.e., avoid predicting a patient will not die within
one year when he actually does.

\red{Put link in the references instead of in the text?}
The data is taken from
\url{https://archive.ics.uci.edu/ml/datasets/Thoracic+Surgery+Data#}
\cite{zieba2013boosted}

\subsection{Desciption of available data}

The data we are working with is about patients who underwent major lung
resections for primary lung cancer in the years from 2007 to 2011. For each
patient we are given information about his diagnosis and effects produced
by the cancer.

The dataset is very limited in the number of instances available as it only contains
data about 470 patients. In addition, the distribution of the predicted class is not quite 
balanced,
since only 70 of the patients died in a one year period after their surgery. This may 
become a problem in some of the prediction models due to the fact that the results 
will be biased towards the bigger class. However, we can suppose that the data has been
collected uniformly and that this proportion is similar to the probability of the real population.

For each patient we have 16 different attributes of which three are numerical, and
the rest is categorical. From those, ten are binary variables. The response attribute is
also binary.

\subsection{Instructions for running the code}
\red{
\begin{itemize}
   % \item Cual debe ser el working directory
    \item La variable que contiene el path que tiene que modificar
    %\item Los resultados ya están calculados en un working space, para ahorrar tiempo.
    \item TODO: implement relative path to data in R or create variable to set this path
\end{itemize}
}

This work comes with the file \textit{code/.RData} containing the imported dataset, all the defined functions and models and the results of all tested models. 
If for any reason the code should be rerun, the proceeding is as follows:\\
To run the code it is necessary to change the working directory of R to the folder \textit{code} which is contained in the archive of this work. In the file \textit{Working\_script.R} all needed R scripts which import the given dataset, do preprocessing and define necessary functions and models are called. The method \textit{getSamples()} creates the test and training samples; with the method \textit{setWeight(value)} the weight of the F2 score could be changed. The call \textit{getResults()} trains and evaluates all the models which takes about 30 minutes on a computer with 8GB RAM and SSD installed. The variable \textit{results} contains a table with all the F2 scores of the models. The variable \textit{f} contains the F2 score for a "model" that always predicts false which we thought an interesting comparison to the calculated models.\\
To see information about the single models there are different variables of the form \textit{knn.res.orig} (KNN model results for the original dataset) for every model. Each of these contains a list of eleven models which can be accessed by \textit{knn.res.orig\$SuperModel[[index]]}, the F2 score \textit{knn.res.orig\$Score} and the confusion matrix \textit{knn.res.orig\$ConfMatrix}.

\subsubsection{Needed packages}

The following R packages have been used in this work:

\begin{itemize}
\item{klaR}
\item{randomForest}
\item{caret}
\item{ROCR}
\end{itemize}